\begin{subsecao}{Bridge}

O Bridge é um jogo pouco conhecido aqui no Brasil mas que é muito jogado em
vários outros países pelo mundo. Usa a mesma dinâmica de vazas dos outros jogos
citados anteriormente e ainda adiciona um contrato* a ser cumprido por uma das
parcerias.

Apesar da sua falta de popularidade no Brasil, uma quantidade significativa de 
veteranes do IME conhecem e jogam o jogo.

O jogo é dividido em duas partes: leilão e carteio. A parte do carteio é bem
parecida com uma Posi no King porém com uma diferença fundamental: as 13 cartas
de um dos jogadores fica à vista, tanto para o seu parceiro quanto para os seus
adversários. Além disso, durante o leilão, várias informações são trocadas entre
as parcerias para tentar se chegar ao melhor número de vazas que podem ser
feitas.

Como já deu pra perceber, o jogo é bastante diferente dos outros e seu
aprendizado é um pouquinho mais complicado, então não vamos explicar nesse Guia
todas as regras, pontuação e convenções utilizadas.

Mas aqui vai uma explicação sobre como esse jogo funciona e algumas de suas regras:
\textbf{O básico}

\begin{itemize}

\item \textbf{Jogadores:} 4 pessoas divididas em duas duplas (Norte-Sul contra Leste-Oeste).
\item \textbf{Baralho:} Usa-se um baralho comum de 52 cartas (sem curingas). Cada jogador 
recebe 13 cartas.
\item \textbf{Hierarquia:} O Ás é a carta mais alta e o 2 a mais baixa. A ordem dos naipes 
(do maior para o menor) é: Espadas ($\spadesuit$), Copas ($\heartsuit$), Ouros ($\diamondsuit$) e Paus ($\clubsuit$).

\end{itemize}

\textbf{Como Funciona o Jogo Uma partida de bridge é dividida em duas fases principais:}

\begin{itemize}

\item \textbf{O Leilão (Bidding):}
Antes de jogar as cartas, os jogadores "apostam" quantas vazas (rodadas) acham que sua 
dupla consegue ganhar e qual naipe será o Trunfo (o naipe dominante) ou se jogarão Sem 
Trunfo (NT).
A aposta vencedora define o Contrato (ex: ganhar 10 vazas com Copas como trunfo).
\item \textbf{O Carteio (The Play):}
A dupla que venceu o leilão tenta cumprir o contrato.
Um jogador da dupla (o Declarante) joga as cartas da sua mão e também as do parceiro 
(o Morto), cujas cartas ficam abertas na mesa.
O jogo acontece em 13 rodadas (vazas). É obrigatório seguir o naipe da carta jogada primeiro. 
Quem jogar a carta mais alta do naipe (ou um trunfo) ganha a vaza.

\end{itemize}

\textbf{Objetivo:}
Se a dupla do Declarante conseguir o número de vazas prometido no contrato, eles ganham pontos.
Se não conseguirem, a dupla adversária ganha pontos.



\end{subsecao}
