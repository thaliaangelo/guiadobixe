\begin{subsecao}{L.E.A.R.N.}

\figurapequenainline{learn}

Você vem acompanhando o rápido avanço dos modelos de inteligência artificial e seus
impactos transformadores em nosso mundo? Está interessado em aprofundar seus
conhecimentos nesse universo fascinante? Então, seja muito bem-vindo ao LEARN.

O LEARN (Liga de Estudos em Aprendizado de Máquina e Redes Neurais) é o 
grupo de extensão focado em aprendizado de máquina do IME. O grupo desenvolve 
diversas atividades tendo como propósito tanto fomentar a pesquisa científica 
quanto capacitar novos pesquisadores e profissionais na área. Fundado em 2023, 
o LEARN consiste de alunos da graduação e pós-graduação do IME que, desde então, 
vêm colaborando no desenvolvimento de diversos projetos pessoais e científicos.

O grupo organiza-se semanalmente em reuniões para a discussão de tópicos voltados 
à área de aprendizado de máquina e para a colaboração no projeto dos integrantes, 
tendo como principais valores a liberdade pessoal e a proatividade. Destacam que 
é uma ótima oportunidade tanto para entender a fundo a área quanto para desenvolver 
habilidades como comunicação, liderança e trabalho em equipe.

Este ano, o grupo irá disponibilizar um Crash Course para a capacitação de novos 
membros, que terá uma duração de cerca de dois meses. Após este período, o aluno 
será capaz de desenvolver projetos pessoais e colaborar em projetos que julgar 
interessantes dos membros do grupo! O processo de admissão de novos membros será 
o Crash Course em si, que não tem restrição nenhuma sobre quem pode se inscrever. 
Contudo, recomenda-se conhecimento prévio em Álgebra Linear, Cálculo I e II e 
Estatística. Tenha certeza que os membros estão dispostos a tirar quaisquer 
dúvidas sobre o processo e animados para ver vocês no curso!

Junte-se ao LEARN e não apenas testemunhe, mas compreenda e influencie o futuro
que está sendo forjado pela inteligência artificial.

begin{description}

    \item[Instagram:] \url{https://www.instagram.com/learnimeusp/}

\end{description}

\end{subsecao}


