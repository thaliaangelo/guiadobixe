\begin{subsecao}{Bisca}
\setlength{\parindent}{0pt}

A Bisca é um jogo de baralho do Espírito Santo que vem ganhando popularidade no
IME devido ao trabalho árduo dos capixabas do instituto. O baralho utilizado é
o mesmo do Truco (sem as cartas 8, 9, 10 e coringas) porém numa ordem bem
diferente.

A ordem crescente de força das cartas é:
\begin{center}
    \large $2, 3, 4, 5, 6, Q, J, K, 7, A$
\end{center}

\bigskip
\noindent\textbf{Funcionamento Básico}

É um jogo de duplas (4 pessoas) que roda em sentido anti-horário. O carteador
embaralha e o jogador à esquerda corta o baralho e puxa o \textbf{trunfo}. O
carteador distribui três cartas para cada jogador.

A mecânica principal gira em torno da \textbf{mão}, que é o conjunto das 4
cartas jogadas numa rodada, e do \textbf{encarte}. Veja alguns conceitos
importantes:
\begin{itemize}
    \item \textbf{Carta Mandante:} A primeira carta jogada na mão define o
    naipe mandante.
    \item \textbf{Encarte:} Jogar uma carta \textbf{do mesmo naipe} da
    mandante, porém \textbf{maior} na sequência.
    \item \textbf{Trunfo:} O naipe definido no início da partida. Qualquer
    carta do naipe de trunfo ganha de cartas de outros naipes (corta).
\end{itemize}

Vence a mão quem jogar a carta mais forte (seja por encarte ou corte). Essa
dupla recolhe as cartas e o jogador que venceu começa a próxima rodada. Todos
compram uma carta nova do monte após cada mão.

\bigskip
\noindent\textbf{Pontuação}

A pontuação da \textbf{partida} de Bisca é feita da seguinte maneira:

\begin{center}
\begin{tabular}{|c|c|}
\hline
\textbf{Carta} & \textbf{Pontos} \\
\hline
$Q$ & $2$ \\
\hline
$J$ & $3$ \\
\hline
$K$ & $4$ \\
\hline
{\color{red} $7$ & \color{red}$10$} \\
\hline
{\color{red} $A$ & \color{red}$11$} \\
\hline
\end{tabular}
\end{center}

As cartas $2,3,4,5,6$ não valem ponto nenhum na partida (são chamadas de "nada"
ou "descarte"). A soma totaliza $120$ pontos; vence a dupla que fizer mais de
$60$.

As cartas em vermelho, ou seja, \color{red}7\normalcolor, \color{red}A\normalcolor,
são o que chamamos de {\color{red}{\textbf{biscas}}}, pois elas são as cartas
que mais valem pontos no jogo.

\bigskip
\noindent\textbf{Fichas e Pontos Extras}

O jogo de Bisca é uma corrida até 4 fichas. Ganhar uma partida (>60 pontos)
vale 1 ficha. Situações especiais valem \textbf{fichas extras}:

\begin{enumerate}
    \item \textbf{Réle (Heling/Reli/...):} O encarte do \textbf{$A$ no $7$ de
    trunfo}. Vale uma ficha extra \textbf{instantaneamente}.

    \item \textbf{Rodar o 7:} Se a primeira carta jogada \textbf{da partida}
    for o $7$ de trunfo e ele rodar a mesa sem levar um réle (ser encartado),
    então a dupla que jogou deve mostrar que não possui o $A$ de trunfo: caso
    não possuam, a dupla ganha uma ficha extra instantaneamente.

    \item \textbf{Capote:} Vencer fazendo \textbf{90 pontos ou mais}. Vale uma
    ficha extra.

    \item \textbf{61 $\times$ 59:} Ganhar pela diferença mínima. Vale uma
    ficha extra.
\end{enumerate}

\textbf{Obs:} Quando o trunfo for \textbf{Copas} $\heartsuit$, todas as fichas
\textbf{extras} valem o \textbf{dobro}.

\bigskip
\noindent\textbf{Regras Importantes}

\begin{enumerate}
    \item O $A$ de trunfo só pode ser jogado após o $7$ de trunfo ter saído
    (sido jogado).
    \item O $7$ de trunfo não pode ser a última carta da mão ("sair de fundo").
    \item \textbf{Exceção:} Se um jogador tiver $7$ e $A$ de trunfo na mão
    \textbf{ao mesmo tempo}, as regras 1 e 2 não se aplicam a ele.
    \item A primeira mão da partida é muda (sem comunicação).
    \item \textbf{Carta Morta:} Carta que cai na mesa com face para cima sem
    intenção de jogo deve permanecer lá e ser jogada obrigatoriamente na vez do
    jogador.
    \item \textbf{Bater o Baralho:} No corte, pode-se "bater" no baralho em vez
    de puxar carta. O trunfo vira Copas $\heartsuit$ no escuro.
\end{enumerate}

Bisca pode parecer muito complicado pra quem nunca jogou, mas não deixe isso te
afastar! As regras ficarão mais naturais conforme se joga, e você vai perceber o
quão legal esse jogo é!

Caso restem dúvidas, basta falar com o Gabriel Flamengo, do BCC, ou com
qualquer capixaba que você encontrar!

\end{subsecao}
